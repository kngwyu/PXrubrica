\documentclass[a4paper]{article}
\usepackage{ifthen,ifxetex,ifluatex}
\ifthenelse{\boolean{xetex}}{
  \usepackage{xeCJK}
  \setCJKmainfont[Scale=0.92]{IPAMincho}
  \setCJKsansfont[Scale=0.92]{IPAGothic}
  \newcommand{\mcfamily}{\rmfamily}
  \newcommand{\gtfamily}{\sffamily}
}{\ifthenelse{\boolean{luatex}}{
  \usepackage{luatexja-fontspec}
  \setmainjfont[Scale=0.92]{IPAMincho}
  \setsansjfont[Scale=0.92]{IPAGothic}
}{}}
\usepackage[scale=0.8]{geometry}
\usepackage{setspace}
\usepackage{pxrubrica}
\newcommand*{\vb}{\symbol{`\|}}
\newcommand*{\Opt}[1]{\texttt{#1}}
\newcommand*{\PKN}[1]{\textsf{#1}}
\newcommand*{\Eg}{E.\,g.\mbox{}}
\newcommand*{\eg}{e.\,g.\mbox{}}
\newcommand*{\ie}{i.\,e.\mbox{}}
\newcommand*{\myfileversion}{1.0}
\newcommand*{\myfiledate}{2011/09/04}
\newsavebox{\myexample}
\rubysetup{f}
\begin{document}

\title{The \PKN{pxrubrica} package}
\author{Takayuki YATO\quad (aka ``ZR'')}
\date{\myfileversion \quad[\myfiledate]}
\maketitle

\setstretch{1.5}

\section{Package Loading}

There is no options available.

\begin{quote}\small\begin{verbatim}
\usepackage{pxrubrica}
\end{verbatim}\end{quote}

\section{Basic Usage}

\subsection{Very Basic}

\begin{itemize}
\item Mono ruby (\Opt{m} option): one ruby block per one kanji.\\
\Eg\quad
\verb+\ruby[m]{鷹}{たか}+ $\rightarrow$ \ruby[m]{鷹}{たか}\quad
\verb+\ruby[m]{鶯}{うぐいす}+ $\rightarrow$ \ruby[m]{鶯}{うぐいす}
\item Group ruby (\Opt{g} option): one ruby block per kanji sequence.\\
\Eg\quad
\verb+\ruby[g]{雲雀}{ひばり}+ $\rightarrow$ \ruby[g]{雲雀}{ひばり}\quad
\verb+\ruby[g]{不如帰}{ほととぎす}+ $\rightarrow$ \ruby[g]{不如帰}{ほととぎす}
\item Jukugo ruby (\Opt{j} option): one ruby block per kanji, but read as whole.\\
\Eg\quad
\verb+\ruby[j]{孔雀}{く|じゃく}+ $\rightarrow$ \ruby[j]{孔雀}{く|じゃく}\quad
\verb+\ruby[j]{七面鳥}{しち|めん|ちょう}+ $\rightarrow$ \ruby[j]{七面鳥}{しち|めん|ちょう}
\item A \verb+|+ symbol in a ruby string denotes the boundary of reading for each kanji
(\ie\ 孔 = く and 雀 = じゃく).
It is not needed in group ruby.
\item Comparison of typeset results:
\begin{quote}\begin{tabular}{ll@{\ }l@{\ }c*3{@{\quad}c}}
Mono & (\verb+\ruby[m]{小鳩}{こ|ばと}+) & $\rightarrow$
  & \ruby[m]{小鳩}{こ|ばと}
  & \ruby[m]{鶺鴒}{せき|れい}
  & \ruby[m]{雷鳥}{らい|ちょう}
  & \ruby[m]{燕}{つばめ}
\\
Group & (\verb+\ruby[g]{小鳩}{こばと}+) & $\rightarrow$
  & \ruby[g]{小鳩}{こばと}
  & \ruby[g]{鶺鴒}{せきれい}
  & \ruby[g]{雷鳥}{らいちょう}
  & \ruby[g]{燕}{つばめ}
\\
Jukugo & (\verb+\ruby[j]{小鳩}{こ|ばと}+) & $\rightarrow$
  & \ruby[j]{小鳩}{こ|ばと}
  & \ruby[j]{鶺鴒}{せき|れい}
  & \ruby[j]{雷鳥}{らい|ちょう}
  & \ruby[j]{燕}{つばめ}
\end{tabular}\end{quote}
Usually When a jukugo has per-character reading then jukugo ruby
(\Opt{j}) is preferred, otherwise (\Opt{g}) is.
If you particularly intend to show the per-character reading
for a jukugo, you might use the \Opt{m} option.
Note that all of \Opt{m}, \Opt{g} and \Opt{j} lead to
the same result for ruby to a single kanji.
\item You can give a default value of option
using the \verb+\rubysetup+ command;
\verb+\rubysetup{g}\ruby{軍鶏}{しゃも}+ is
equivalent to \verb+\ruby[g]{軍鶏}{しゃも}+.
The ``default of default'' is \Opt{|cjPeF|}.
\end{itemize}

\subsection{Intrusion/Protrusion}

\begin{itemize}
\item Control of ruby intrusion:
\begin{center}\begin{tabular}{ll@{\ }l@{\ }c*2{@{\quad}c}}
No intrusion & \verb+この\ruby[|-|]{鵲}{かささぎ}の+ & $\rightarrow$
 & この\ruby[|-|]{鵲}{かささぎ}の
 & この\ruby[|-|]{鸛}{こうのとり}の
 & この\ruby[|-|]{鵜}{う}の
\\
Small intrusion & \verb+この\ruby[(-)]{鵲}{かささぎ}の+ & $\rightarrow$
 & この\ruby[(-)]{鵲}{かささぎ}の
 & この\ruby[(-)]{鸛}{こうのとり}の
 & この\ruby[(-)]{鵜}{う}の
\\
Big intrusion & \verb+この\ruby[<->]{鵲}{かささぎ}の+ & $\rightarrow$
 & この\ruby[<->]{鵲}{かささぎ}の
 & この\ruby[<->]{鸛}{こうのとり}の
 & この\ruby[<->]{鵜}{う}の
\end{tabular}\end{center}
\item If you require ``ruby output may intrude to kana but not to kanji,''
then you might want to do ``\verb+この\ruby[<-|]{鵲}{かささぎ}等+''
to get ``この\ruby[<-|]{鵲}{かささぎ}等.''
\item To specify a basic mode (\Opt{m}/\Opt{g}/\Opt{j})
and intrusion at a time, you can use option strings
such as \Opt{|g|} and \Opt{|m>}.
In fact, the symbol \Opt{-} works as placeholder
for basic mode and means the use of default value.
\item Control of ruby protrusion:
the string \Opt{||} supresses the protrusion.
\begin{quote}
\begin{lrbox}{\myexample}
\small$\leftarrow$ \verb+\ruby[||->]{雀}{すずめ}+
\end{lrbox}
\fbox{\parbox{.42\linewidth}{%
\ruby[||->]{雀}{すずめ}の…
\quad \usebox{\myexample}%
\rule{0pt}{12pt}\\
インコの
}}\quad vs.\quad
\begin{lrbox}{\myexample}
\small$\leftarrow$ \verb+\ruby[|->]{雀}{すずめ}+
\end{lrbox}
\fbox{\parbox{.42\linewidth}{%
\ruby[|->]{雀}{すずめ}の…
\quad \usebox{\myexample}%
\rule{0pt}{12pt}\\
インコの
}}
\end{quote}
\end{itemize}

\subsection{More Commands}

\begin{itemize}
\item \verb+\aruby+: attaches ruby to an alphabet (non-CJK) string.
\par\noindent\Eg\quad
\verb+\aruby{Get out}{ゲラウッ}!+ $\rightarrow$
  \aruby{Get out}{ゲラウッ}!
\item \verb+\rubyfontsetup+: specifies the font used for ruby output.
For example, when you want to attach mincho-font ruby
to gothic-font kanji string you can do as follows:
\par\noindent
{\small
\verb+\rubyfontsetup{\mcfamily}この{\gtfamily \ruby[j]{明朝体}{みん|ちょう|たい}}+}
$\rightarrow$
{\rubyfontsetup{\mcfamily}この{\gtfamily \ruby[j]{明朝体}{みん|ちょう|たい}}}
\end{itemize}

\end{document}
